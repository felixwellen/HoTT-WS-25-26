\subsection{$n$-Typen}

Wir sind bereits auf den Begriff der Kontrahierbarkeit gestoßen.
Die kontrahierbaren Typen bilden die unterste Stufe einer Hierarchie
von Typen mit zunehmen komplizierteren Identitätstypen.
Die nächste Stufe in dieser Hierarchie, der sogenannten $n$-Typen bilden die Aussagen.

\begin{definition}
  Ein Typ $A$ heißt \begriff{Aussage}\index{$\isProp(A)$}, wenn der folgende Typ bewohnt ist
  \[
    \isProp(A)\colonequiv (x\ y : A)\to x=y\rlap{.}
  \]
\end{definition}

Kontrahierbare Typen sind stets Aussagen.
Durch eine ``Zusatzbedingung'' wird jede Aussage kontrahierbar und tatsächlich haben wir auch die sehr nützliche Umkehrung diese Prinzips.

\begin{bemerkung}
  Sei $A:\mU$.
  \begin{enumerate}[(a)]
  \item Wenn $A$ kontrahierbar ist, dann ist $A$ eine Aussage.
  \item Wenn $A$ eine Aussage ist und wir $a:A$ haben, dann ist $A$ kontrahierbar.
  \item Wenn wir eine Funktion des Typs $A\to \isContr(A)$ haben, dann ist $A$ bereits eine Aussage. 
  \end{enumerate}
\end{bemerkung}

\begin{beweis}
  Manche als Übung, andere hier?
\end{beweis}

Eine Aussage ist auch dadurch charakterisiert, dass alle Gleichheitstypen kontrahierbar sind \Cref{prop-id-contractible}.
Dafür benötigen wir noch eine verallgemeinerung des Anwendens von Funktionen auf Gleichheiten:

\begin{definition}
  Seien $B:A\to \mU$ und $f:(z:A)\to B(z)$.
  Für $x,y:A$ gibt es\index{$\apd_f$}
  \begin{align*}
    &\apd_f : (p:x=y) \to \transp_B(p)(f(x))=f(y) \\
    &\apd_f(\refl)\colonequiv \refl
  \end{align*}
\end{definition}

\begin{lemma}
  \label{prop-id-contractible}
  Sei $A$ eine Aussage, dann haben wir
  \[
    (x\ y :A)\to \isContr(x=y)\rlap{.}
  \]
\end{lemma}

\begin{beweis}
  Sei $h:(x\ y : A)\to x=y$ ein Zeuge, dass $A$ eine Aussage ist.
  Wir werden zunächst zeigen, dass wir $(x\ y : A)\to \isProp(x=y)$ haben.
  Seien dazu $x,y:A$ und $p,q:x=y$. Dann wenden wir $h(x):(z:A)\to x=z$ an auf $p$ und $q$:
  \[
  \apd_{h(x)}(p):\transp_{z\mapsto x=z}(p)(h(x,x))=h(x,y)\text{ und }
  \apd_{h(x)}(q):\transp_{z\mapsto x=z}(q)(h(x,x))=h(x,y)
  \]
  Nach \Cref{transport-gleichheitstyp} haben wir also:
  \[
    h(x,x)\kon p=h(x,y)=h(x,x)\kon q
  \]
  und (per Gleichheitsinduktion können) wir kürzen: $p=q$.

  Nun ist $x=y$ stets eine bewohnte Aussage und damit kontrahierbar.
\end{beweis}

\begin{definition}
  Sei $A$ ein Typ und $\N_{-2}$ ein induktiv definierter Typ,
  mit Konstruktoren $-2$ und $S:\N_{-2}\to\N_{-2}$\index{$\N_{-2}$}.
  Für $n:\N_{-2}$ definieren wir induktiv $\isNType{n}(A)$\index{$\isNType{n}(A)$}:
  \begin{align*}
    \isNType{(-2)}(A)&\colonequiv \isContr(A) \\
    \isNType{(n+1)}(A)&\colonequiv (x\ y : A)\to \isNType{n}(x=y)
  \end{align*}
  $0$-Typen werden auch \begriff{Mengen} genannt und $\isNType{0}(A)$ wird auch als $\isSet(A)$\index{$\isSet(A)$} geschrieben.
\end{definition}

\begin{beispiel}
  Wir haben bereits gesehen, dass $\eins$ kontrahierbar ist.
  Weiter sind $\leer$ und $\eins$ Aussagen.
\end{beispiel}

Für weitere Beispiele und um zu sehen, dass nicht alle Typen Aussagen sind, sollten wir erst folgende Definition machen:

\begin{definition}
  Sei $A$ ein Typ.\index{$\neg A$}
  \[
    \neg A\colonequiv A\to\leer
    \]
    Im Fall $A\equiv x=y$ schreiben wir auch $x\neq y$ statt $\neg(x=y)$.
\end{definition}

\begin{bemerkung}
  Im Typ $\zwei$ gilt: $0_\zwei\neq 1_\zwei$. Und damit auch $\neg(\isProp(\zwei))$.
\end{bemerkung}
\begin{beweis}
  Wir definieren $f:\zwei\to\mU$ durch:
  \[
    f(0_\zwei)\colonequiv \leer\text{ und }f(1_\zwei)\colonequiv \eins
    \]
    Für eine Gleichheit $0_\zwei=1_\zwei$ bekommen wir durch transport eine Funktion $\eins\to\leer$.
\end{beweis}

\subsection{Äquivalenzen}

Wir sind bereits auf Abbildungen $f:A\to B$ und $g:B\to A$ gestoßen, die \emph{zueinander inverse} sind, also sodass die Typen
\[
  (x:A)\to g(f(x))=x\quad \text{ und }\quad (y:B)\to f(g(y))=y 
\]
bewohnt sind.
In der Mengenlehre würden analoge Daten zeigen, dass $f$ eine Bijektion ist.
Das Wort ``Bijektion'' wollen wir nur für - die noch zu definierenden - Mengen verwenden.

Oben fragen wir nach \emph{punktweisen Gleichheiten}, etwa zwischen $g\circ f$ und $\id_A$ und
nicht direkt nach einem Term in $g\circ f = \id_A$.
Tatsächlich lässt sich aus den bisher eingeführten Regeln keine Funktion folgenden Typs konstruieren - für $f,g:A\to B$:
\[
  ((x:A)\to f(x)=g(x))\to f=g
\]
Eine solche Funktion werden wir später durch das Funktionsextensionalitätsaxiom erzwingen,
aber nun werden wir erstmal nur mit punktweisen Gleichheiten und nicht Gleichheiten von Funktionen arbeiten, die wir auch \emph{Homotopien} nennen.

\begin{definition}
  Seien $A,B$ Typen und $f,g:A\to B$.
  Der Typ der \begriff{Homotopien}\index{$f\sim g$}
  oder \begriff{punktweisen Gleichheiten} zwischen $f$ und $g$
  ist gegeben durch
  \[
    (f\sim g)\colonequiv( (x:A)\to f(x)=g(x))\rlap{.}
  \]
\end{definition}

Durch punktweises Andwenden unserer Konstruktionen für die Gleichheit,
bekommen wir eine sehr ähnliche Struktur auf den Homotopien:

\begin{definition}
  Seien $A,B$ Typen und $f,g,h:A\to B$.
  \begin{enumerate}[(a)]
  \item Für jedes $f$ gibt es stets die Homotopie $\refl_f:f\sim f$ gegeben durch $x\mapsto \refl_{f(x)}$.
  \item Für eine Homotopie $H:f\sim g$ gibt es stets die Inverse Homotopie $H^{-1}:g\sim f$ gegeben durch $H^{-1}(x)\colonequiv H(x)^{-1}$.
  \item Für $H:f\sim g$ und $K:g\sim h$ gibt es $H\kon K\colonequiv (x:A)\to H(x)\kon K(x)$.
  \end{enumerate}
\end{definition}

Wir werden nun zunächst den Typ der quasi-inversen zu einer gegebenen Funktion definieren
- dabei gibt es einen subtilen Unterschied zum Äquivalenzbegriff auf den wir noch eingehen werden.

\begin{definition}
  Seien $A,B$ Typen und $f:A\to B$. Der Typ der \begriff{quasi-Inversen}\index{$\mathrm{qInv}(f)$} von $f$ ist gegeben durch
  \[
    \mathrm{qInv}(f)\colonequiv (g:B\to A)\times (f\circ g\sim \id_A\times g\circ f\sim \id_B)\rlap{.}
  \]
\end{definition}

Es wird sich noch herausstellen, dass der folgende Typ der Äquivalenz-Nachweise zu einer gegebenen Abbildung im allgemeinen verschieden ist:

\begin{definition}
  Seien $A,B$ Typen und $f:A\to B$. \index{$\isEquiv(f)$}Der folgende Typ steht für die Eigenschaft von $f$ eine Äquivalenz zu sein:
  \[
    \isEquiv(f) \colonequiv ((g:B\to A)\times g\circ f\sim\id_A )\times ((h:B\to A)\times f\circ h\sim\id_B)\rlap{.}
  \]
\end{definition}
