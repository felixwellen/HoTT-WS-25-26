% latexmk -pdf -pvc blatt1.tex
\documentclass{hott-übung}

\begin{document}

\setcounter{blattnummer}{1}
\setcounter{aufgabennummer}{0}

\blatt

\aufgabe{Assoziativität der Komposition}
  Seien $f:A\to B$, $g:B\to C$ und $h:C\to D$ Funktionen.
  Begründe wie in der Vorlesung\footnote{Bemerkung 1.1.5 (a) im Skript(\url{https://felix-cherubini.de/HoTT.pdf})} dass gilt $h\circ(g\circ f)\equiv (h\circ g)\circ f$.

\aufgabe{Konstante Funktionen}
  Seien $A,B$ Typen und $b:B$.
  \begin{enumerate}[(a)]
    \item Definiere die konstante Funktion $\mathrm{const}_{A,B,b}:A\to B$.
    \item Sei $f:A'\to A$ eine Funktion.
      Zeige $\mathrm{const}_{A,B,b}\circ f \equiv \mathrm{const}_{A',B,b}$.
  \end{enumerate}

\aufgabe{}
\begin{enumerate}[(a)]
\item Zeige $1+1\equiv 2$. Verwende dafür nur die Regeln für $\N$ und die Definition von $1$, $2$ und die Eliminator-basierte Definition von $+$.
\item Definiere eine Abbildung $m:\N\to\N\to\N$, die der bekannten Multiplikation natürlicher Zahlen entspricht. Die Funktion ``$+$'' darf dabei verwendet werden.
\end{enumerate}
  
\aufgabe{Pattern-Matching}
  Erkläre wie die folgenden Pattern-Matching-Definition einer Funktion $f:\N\to\N\to\N$ in eine
  Definition die nur Eliminatoren, aber kein Pattern-Matching verwendet übersetzt werden kann.
  \begin{align*}
    f(0)(b) &\colonequiv S(b) \\
    f(S(a))(0) &\colonequiv f(a)(S(0)) \\
    f(S(a))(S(b)) &\colonequiv f(a)(f(S(a))(b))
  \end{align*}
\end{document}

