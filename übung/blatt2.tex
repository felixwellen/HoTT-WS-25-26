% latexmk -pdf -pvc blatt2.tex
\documentclass{hott-übung}

\begin{document}

\setcounter{blattnummer}{2}
\setcounter{aufgabennummer}{0}

\blatt

\aufgabe{}
\begin{enumerate}[(a)]
\item Analog zum Typ $\eins$ aus der Vorlesung gibt es auch einen induktiven Typ $\zwei$, mit zwei Konstruktoren, die jeweils keine Argumente haben.
Finde passende Regeln für diesen Typ, insbesondere einen Eliminator.\\
\emph{Bemerkung:} Wir hatten diesen Typ in Agda als ``Bool'' kennen gelernt. 
\item Konstruiere Abbildungen $\varphi:\eins\sqcup \eins\to \zwei$ und $\psi:\zwei\to \eins\sqcup \eins$
zusammen mit Termen in
\[
  (x:\eins\sqcup \eins)\to \psi (\varphi(x))=x\quad \text{ und }\quad (x:\zwei)\to \varphi(\psi(x))=x
\]
\end{enumerate}

\aufgabe{}
Zeige, dass die Addition $+$ auf $\N$ assoziativ ist:
Für alle $n,k,l:\N$ haben wir $(n+k)+l=_\N n+(k+l)$.
Ganz formal ausgedrückt ist die Aufgabe also, einen Term des folgenden Typs zu konstruieren.
\[
  (n,k,l:\N) \to (n+k)+l=_\N n+(k+l)
\]

\aufgabe{}
Sei $A$ ein Typ. Zeige:
\begin{enumerate}[(a)]
\item Für alle $x,y : A$ und $ p:x=_A y$ haben wir $ p\kon p^{-1}=\refl_x= p^{-1}\kon p$.
\item Für alle $x,y,z : A$ und $ p:x=_Ay$, $ q:y=_Az$, $ r:z=_Au$ gilt
   $( p\kon q)\kon r= p\kon( q\kon r)$.
\end{enumerate}

\aufgabe{}
Seien $A$ ein Typ und $B(x)$ ein Typ für $x:A$.
Finde für beliebige $x,y:A$ und $p:x=y$ eine Funktion $\psi_{x,y,p}:B(y)\to B(x)$
sodass 
\[
  (b:B(x))\to \psi_{x,y,p} (\transp_B(p)(b))=b\rlap{.}
\]

\bonus{}
  Sei \(A\) ein Typ, $a,b,c,d,e:A$ Terme und $p:a=b$, $q:b=c$, $r:c=d$, $s:d=e$ Gleichheiten.
  Verwende den in Aufgabe 3 (b) konstruierten \emph{Assoziator}
  \[\ass:(p:x=y)\to (q:y=z)\to (r:z=w) \to (p\kon q)\kon r = p \kon (q\kon r)\]
  um die 5 Gleichheiten $\alpha_1,\dots,\alpha_5$ im sogenannten \emph{MacLane-Pentagon} zu konstruieren:
  \begin{equation*}
    \begin{tikzcd}[column sep={between origins,1.4cm},row sep={between origins,1.5cm}]
      & (p \kon (q \kon r)) \kon s
      \arrow[rr,equal,"\alpha_2"]
      && p \kon ((q \kon r) \kon s)
      \arrow[dr,equal,"\alpha_3"]
      &
      \\
      ((p \kon q) \kon r) \kon s
      \arrow[ur,equal,"\alpha_1"]
      \arrow[drr,equal,"\alpha_4"']
      &&&& p \kon (q \kon (r \kon s))
      \\
      && (p \kon q) \kon (r \kon s)
      \arrow[urr,equal,"\alpha_5"']
      &&
    \end{tikzcd}
  \end{equation*}
  Zeige anschließend, dass $(\alpha_1\kon\alpha_2)\kon\alpha_3 = \alpha_4\kon\alpha_5$ ist.

\end{document}
